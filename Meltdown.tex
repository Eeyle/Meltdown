\documentclass[a4paper]{article}

%% Language and font encodings
\usepackage[english]{babel}
\usepackage[utf8x]{inputenc}
\usepackage[T1]{fontenc}

%% Sets page size and margins
\usepackage[a4paper, top=3cm, bottom=2cm, left=3cm, right=3cm, marginparwidth=2cm]{geometry}

%% Useful packages
\usepackage{amsmath}
\usepackage{graphicx}
\graphicspath{ {./pic/} }
\usepackage[colorinlistoftodos]{todonotes}
\usepackage[colorlinks=true, allcolors=blue]{hyperref}

% added by me
%% Multiple columns
\usepackage{multicol}
%% Pretty Tables
\usepackage{booktabs}
%% Extended column definitions
\usepackage{array}
%% Full Page Graphics
\usepackage{pdfpages}
%% No separation between elements of lists
\usepackage{enumitem}
\setlist{nosep}
%% Include links to websites
\usepackage{hyperref}
%% Generate the degree symbol
\usepackage{gensymb}
%% Drawing connections between nodes of graph
\usepackage{tikz}
\usetikzlibrary{angles, quotes}
%% Multi-line comments
\usepackage{verbatim}

% Dashes not dots
\renewcommand\labelitemi{---}

\title{Meltdown \\ {\large - GM Guide -}}
\author{Eeyle}
\date{}

\begin{document}

\maketitle

\begin{abstract}

A science fiction RPG where the central gameplay involves fixing things and using futuristic weaponry creatively. Players are equipped with futuristic weaponry to be used in combat. Throughout the campaign they solve several mysteries based around modern scientific research and they'll have to use their equipment in new and creative ways in noncombat engagements.

%A science fiction RPG centered around realistic space combat, except that the players hardly participate in the combat itself at all. The campaign seeks to answer how it would feel to really be on a spaceship that was falling apart, and what this kind of intense and dangerous situation can drive people to do. Instead of leading the combat, the players will witness their ship breaking down around them as shots fly and engines roar. The players must understand their particular systems of the ship well enough to identify any problems, come up with solutions to them, and then fix those problems. The players will not be valiant heroes nor mystery investigators, instead they will be average mechanics, plumbers, and pilots who are forced to fix extreme problems, in complicated subsystems, under dangerous circumstances.

\end{abstract}

{
  \hypersetup{linkcolor=black}
  \setcounter{tocdepth}{2}
  \tableofcontents
}

\section{Preface} \label{preface}

\subsection{Notation} \label{preface_notation}
Typical D\&D notation will be used, with some shorthand added. 
\begin{itemize}
\item The \textit{GM} is the game master and controls the game as a narrator.
\item \textit{1d6} means roll one 6-sided die.
% \item \textit{1d4 ? 1. ... 2. ... 3. ... 4. ...} means roll a 4-sided die, and if it comes up 1 then follow the sentence after 1 (whatever is in the first ellipsis), and likewise for 2, 3, and 4.
\end{itemize}

% This is a general form to create dice outcome charts. Its first argument should be the dice used. The next two arguments form a pair, the first being which die outcomes and the second what happens on those outcomes. Each two arguments after that are optional, and form pairs, so that up to four ranges of die outcomes can be represented.
% For example, #1 could be 1d12. #2 and #3 say 1-4. Something happens. Then #4 and #5 say 5-6. Something else happens. Then #6 and #7 say 7-12. A third different thing happens. #8 and #9 are left blank. This gives three die ranges and three outcomes, depending on what you roll on 1d12.
\def\qeq#1#2#3#4#5#6#7#8#9{#1 ? 
\vspace*{-0.4cm} \begin{enumerate}[leftmargin=2cm]
\item [#2] #3 

\if#4% empty
\else
\item [#4] #5
\fi

\if#6% empty
\else
\item [#6] #7
\fi

\if#8% empty
\else
\item [#8] #9
\fi
\end{enumerate}}

% The physical and heating damage outcomes.
\def\pbhw#1#2{
\begin{tabular}[t]{r p{14cm}}
\textit{Physical} - & #1 \\
\textit{Heating} - & #2 \\
\end{tabular}}


\section{Campaign}

\subsection{Fusion Reactor}

\subsection{Quantum Computer}

\subsection{Gravitational Wave Interferometer}

\subsection{Neutrino Detector}

\subsection{Radiotelescope Array}

\subsection{Particle Accelerator}


\section{Engagements}

\subsection{Outside station chasing on foot}

\subsection{Inside station being chased on foot}

\subsection{Space chasing in fighters}

\subsection{Ground chasing in large vehicle}

\subsection{Ground being chased in small vehicles}

\subsection{Space being chased in large ship}


\section{Combats}

\section{Locations}

\subsection{The Magnificent Station above Earth}

\subsection{The Children about Jupiter}

\subsection{The Settlement of Mars}


\section{Upgrades}

Every class is allowed a number of upgrades throughout the campaign. The way in which you award upgrades is entirely up to you. I will award them after one or two completed mysteries depending on the pacing of the game. Players will get a normal upgrade every time and a special upgrade every other time for a total of four normal and two special upgrades to both weaponry and equipment. 

When the time comes to upgrade, ask each player in what way they want to upgrade their weapon and their equipment. This can be in any way the player wishes. Should the player simply want to improve the damage of their weapon, that is encouraged as long as they think about what components specifically they would need to upgrade in order to do so. If they want to upgrade something more creative, it is up to the GM's discretion.

Below is a few suggested upgrades that fit within the balance of the game and the classes. Feel free to avoid these or modify them. These are split into regular upgrades, which can be taken any number of times (provided you come up with a new way of making the upgrade every time), and special upgrades which should only be taken once.

\subsection{Arcmage Upgrades}

Normal Tesla coil upgrades

\begin{itemize}
\item +1 damage per attack. The player should upgrade something to do with the energy being sent to the coil, meaning a better set of capacitors, less resistant electrical components, or something of the like.
\item +1 range (up to a maximum of 5 range for balance reasons). The player should upgrade something about the voltage of the coil, so better transformer coils or a better tuned set of circuits. 
\end{itemize}

Special Tesla coil upgrades

\begin{itemize}
\item Instead of attacking, you can give up any number of attacks. If you do, add +N damage to all attacks next turn, where N is the number of attacks you gave up.
\item Instead of making two separate attacks, you can combine your two attacks into one. If you do, it hits a single random component of either your choice of the thermal or electrical system. This attack loses -2 damage each time it leaps to another enemy instead of -1.
\item You now have 3 attacks per turn but take -3 damage to each attack.
\item You may move backwards one meter every time you make an attack.
\item Every time an attack leaps to a relay, add either +1 damage or +1 range to it.
\item The last enemy hit by every attack you make takes an additional +1 damage per range left over in the attack.
\end{itemize}

Normal equipment upgrades

\begin{itemize}
\item -1 size to any component. The player should describe how they are miniaturizing the component and the GM may ask for clarification about complications.
\item +25\% max heat to any component. The player should find some way of insulating the component without getting in the way of other parts.
\item +1 max heat to all components. The player should describe what kind of insulation they apply to their entire system.
\item +1 max phys to all components. The player should detail exactly how they are reinforcing the different components.
\item -1 coil heat flow. How can more thermal piping get into the coil without causing disruption to the circuitry?
\item -1 robes heat flow. How could a player induce more evaporation or more flow throughout the robes?
\item -1 min thermal fluid (absolute minimum 1). What efficiencies is the player making that allows less fluid to do more work? 
\end{itemize}

Special equipment upgrades

\begin{itemize}
\item -1 other heat flow. The player should describe a major upgrade they make to allow the entirety of the thermal system to function more smoothly.
\item Remove a component from ever being hit. Ideally this is used because a player is tired of one component getting hit constantly, though it can be preventative as well. The player should come up with some way that they might hide this component deep away in the recesses of their outfit.
\item Add a decoy component of size 3 to a system. The player should come up with a shape and function that it might make, that's completely false. You may also elect this as an upgrade if the player simply wants to add armor to the outside of themselves.
\item +2 total thermal fluid. The player should justify why they can carry more weight in water and how it fits into their outfit. 
\item +1 movement speed. The player can describe themselves removing weight and making the entire contraption lighter, or replacing components with lighter counterparts. 
\end{itemize}



















\end{document}